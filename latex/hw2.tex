\documentclass[11pt]{article}
\usepackage{fullpage}
\usepackage{graphicx}
\usepackage{hyperref}
\begin{document}

\title{Homework 2 -- Monte Carlo Path Tracing}
\author{Gabe Fierro and Graham Tremper}
\date{\today}
\maketitle

\section{Introduction}

The code was compiled using GLM 0.9.4.1 on Mac OSX 10.7.4 and 10.8.2. To compile, run \verb`make` from
the \verb`PathTracing` directory. The test scenes are included in the \verb`testscenes` directory. The
compiled binary, \verb`raytracer` accepts 1 mandatory argument which is the path to the test scene, and
1 optional argument \verb`-f X`, where \verb`X` is the number of samples you wish to send at the scene.
For example

\begin{center}
	\verb`prompt> ./raytracer testscenes/cornell_box.test`
	\verb`prompt> ./raytracer -f 1000 testscenes/cornell_box.test`
\end{center}

If \verb`raytracer` is run without the optional \verb`-f` option, it enters an interactive mode. Pressing \verb`l`
will send a round of samples at the scene and render the result. Pressing \verb`f` will send 10000 samples
at the scene, and render after each result. Pressing \verb`s` will save the scene as a png file.

\section{Ray Tracing}

The project was built upon a basic ray tracer programmed for CS184, which supported rendering basic Phong models -- diffuse, specular and shiny surfaces. Triangle and sphere primitives are supported, and
a KD-tree is used as an accelerating structure.

\section{Global Illumination}

\end{document}
